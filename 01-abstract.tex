\begin{abstract}

Evaluating whether vision–language models (VLMs) reason consistently across representations is challenging because modality comparisons are typically confounded by task differences and asymmetric information. We introduce \textbf{SEAM}, a benchmark that pairs semantically equivalent inputs across four domains with existing standardized textual and visual notations. By employing distinct notation systems across modalities, in contrast to OCR-based image-text pairing, SEAM provides a rigorous comparative assessment of the textual-symbolic and visual-spatial reasoning capabilities of VLMs. Across 21 contemporary models, we observe systematic modality imbalance: vision frequently lags language in overall performance, despite the problems containing semantically equivalent information, and cross-modal agreement is relatively low. Our error analysis reveals two main drivers: textual perception failures from tokenization in domain notations and visual perception failures that induce hallucinations. We also show that our results are largely robust to visual transformations. SEAM establishes a controlled, semantically equivalent setting for measuring and improving modality-agnostic reasoning. The \textbf{SEAM} benchmark dataset and code are available at \url{https://github.com/CSSLab/SEAM}.


\end{abstract}

